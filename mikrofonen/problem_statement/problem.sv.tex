\problemname{Mikrofonen}
Det visar sig att trumman överlevde Mattias. Kanske kommer namnet "The Destroyer" från att folks trumhinnor spräcks av volymen.
Vi installerade en mikrofon i musikrummet som avläser ljudstyrkan i decibel en gång i sekunden under $N$ sekunder. 
Trumhinnan spräcks vid 165 decibel, men mikrofonen är inte helt pålitlig. Ibland kan den hävda helt osannolika värden.

Mer specifikt accepterar vi en ljudstyrka $V$ om $V>=l$ och $V<=h$ för två heltal l och h. Du undrar nu vad medelvärdet på volymen var.


\section*{Indata}
Den första raden innehåller tre heltal: $1\leq N \leq 10^4$, $20 \leq l \leq 300$ och $20 \leq h \leq 300$  -- antalet personer, lägsta rimliga och högsta rimliga volymen. 
Därefter följer $N$ heltal $-10^9 \leq V \leq 10^9$, de avmätta volymerna.
Det är garanterat att minst ett tal är inom gränserna.

\section*{Utdata}
Skriv ut ett decimaltal -- medelvärdet på volymen över alla rimliga värden.
Svaret kommer accepteras om det har ett relativt eller absolut fel om högst $10^{-5}$.
Dvs, om ditt svar är $a$ och det korrekta svaret är $b$, så accepteras ditt svar om
antingen $|a-b| \le 10^{-5}$ eller $\frac{|a-b|}{|b|} \le 10^{-5}$.

\section*{Poängsättning}
Din lösning kommer att testas på en mängd testfallsgrupper.
För att få poäng för en grupp så måste du klara alla testfall i gruppen.

\noindent
\begin{tabular}{| l | l | p{12cm} |}
  \hline
  Grupp & Poängvärde & Gränser \\ \hline
  $1$   & $15$       & $N=1$ \\ \hline
  $1$   & $30$       & $l \leq V \leq h$ för alla v \\ \hline
  $3$   & $55$       & Inga ytterligare begränsningar \\ \hline
\end{tabular}

\section*{Förklaring av exempelfall}
I första exemplet ligger 230 utanför spannet av rimliga mätvärden, medelvärdet blir då $\frac{50+72+80}{3} = 67.333$.
I Andra exemplet ligger endast 40 inom intervallet, så svaret blir 40.