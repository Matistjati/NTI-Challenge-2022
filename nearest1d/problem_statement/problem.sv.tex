\problemname{Jakob WIDEbrant}
Efter att ha rättat otaliga gymnasiearbeten och tillämpad programmeringsprojekt är Jakob äntligen på semester! Han står nu skidbacken wide brant- den breda branten.
Precis som han ska börja åka märker han till sin förfäran att han inte kan slänga- kanske han inte borde gett F till han där som alltid kommer med en kristallkula.
Wide brant är ett $R*C$ stort rutnät, där en . på en ruta betyder att det är tomt och \# betyder att det finns ett träd. 
Han vill åtminstone göra det bästa av situationen och undrar därför hur många unika vägar han kan åka från toppen av backen till botten utan att krocka i ett träd.

\section*{Indata}
Den första raden innehåller två heltal: $1\leq R \leq 1000$ och $1 \leq C \leq 1000$ -- antalet rader och kolumner. Det är även garanterat att $R \leq C/3$, annars hade det inte varit en bred backe.
Därefter följer $R$ rader, vardera $C$ tecken, som beskriver storms målning. 
Efter en tom rad följer $R$ rader med $C$ tecken som beskriver storms originella bild. 

\section*{Utdata}
Skriv ut ett heltal -- antalet unika vägar som jakob kan åka från toppen till botten utan att krocka i ett träd.

\section*{Poängsättning}
Din lösning kommer att testas på en mängd testfallsgrupper.
För att få poäng för en grupp så måste du klara alla testfall i gruppen.

\noindent
\begin{tabular}{| l | l | p{12cm} |}
  \hline
  Grupp & Poängvärde & Gränser \\ \hline
  $1$   & $10$       & Det finns inga träd \\ \hline
  $2$   & $20$       & $R=1$ \\ \hline
  $4$   & $70$       & Inga ytterligare begränsningar  \\ \hline
\end{tabular}

\section*{Förklaring av exempelfall}
I första exemplet finns det 3 lediga vägar: vid kolumn 1, 4 och 6.
Andra exemplet uppfyller kravet i testgrupp 3. I denna finns det 6 lediga väger, de första 6 kolumnerna.