\problemname{Flygande fika}
Det är en helt vanlig morgon i fikaklubben, dålig musik, roliga skämt (amogus), ända tills Alfred råkar snubbla med kakasken och alla $M$ kakor flyger rakt fram!
Tack vare NTI:s fenomenala utbildning beräknar de $N$ eleverna direkt var kakorna kommer landa. Det råkar vara så att alla i klassrummet står precis på linjen där alla kakorna landar.
För att inte missa $5$-sekundersregeln undrar nu varje person i klassrummet vilken kaka som kommer landa närmast de. 

Mer specifikt kommer alla $N$ person att stå på samma linje som de $M$ kakorna landar på, denna är numrerad $0$ till $L$.


\section*{Indata}
Den första raden innehåller två heltal: $1\leq N \leq 10^5$ och $1 \leq M \leq 10^5$ -- antalet personer och antalet kakor. 
Därefter följer $N$ heltal $0 \leq P \leq L$, positionen på alla personer.
Nästa rad innehåller $M$ heltal $0 \leq K \leq L$, positionen där kakorna landar.
Personer och kakor är sorterade från lägst till störst och det kan inte finnas flera personer eller flera kakor på samma ställe.
Det är garanterat att $L \leq 10^9$.

\section*{Utdata}
Skriv ut $N$ heltal -- indexet på kakan som landar närmast varje person. Om det finns flera kakor som landar lika nära kan du skriva ut vilken som helst.

\section*{Poängsättning}
Din lösning kommer att testas på en mängd testfallsgrupper.
För att få poäng för en grupp så måste du klara alla testfall i gruppen.

\noindent
\begin{tabular}{| l | l | p{12cm} |}
  \hline
  Grupp & Poängvärde & Gränser \\ \hline
  $1$   & $10$       & Alla elever har mindre x-koordinat är kakorna \\ \hline
  $2$   & $20$       & $N=1$ \\ \hline
  $3$   & $30$       & $L \leq 5\cdot 10^5$ \\ \hline
  $4$   & $40$       & Inga ytterligare begränsningar  \\ \hline
\end{tabular}

\section*{Förklaring av exempelfall}
I första exemplet är kakan med index $0$ närmast första personen, kakorna med index $0$ och $1$ lika nära person andra personen och kakan med index $2$ närmast den tredje.
I Andra exemplet är kakan med index $0$ närmast de två första personen och kakan med index $1$ närmast den tredje.
