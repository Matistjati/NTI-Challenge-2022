\problemname{Matkortet}
Linnea skulle till baguetterian för att köpa pannkakor, men blipp-funktionen på kortläsaren funkar inte.
Hon behöver därför använda sin $4$-siffriga kod, men hon kommor bara ihåg en del av siffrorna.
Nu kommer Lisa in, och som om Linnea vore i en matteuppgift frågar Lisa henne vad sannolikheten är att hon gissar resten av koden rätt.   


\section*{Indata}
Den första raden innehåller fyra bokstäver, koden till matkortet. Om hon glömt en siffra är den . istället för en siffra mellan $0$ till $9$.

\section*{Utdata}
Skriv ut ett decimaltal -- sannolikheten att Linnea gissar resten av koden korrekt (mellan $0$ och $1$).

\section*{Poängsättning}
Din lösning kommer att testas på en mängd testfallsgrupper.
För att få poäng för en grupp så måste du klara alla testfall i gruppen.
Svaret kommer accepteras om det har ett relativt eller absolut fel om högst $10^{-5}$.
Dvs, om ditt svar är $a$ och det korrekta svaret är $b$, så accepteras ditt svar om
antingen $|a-b| \le 10^{-5}$ eller $\frac{|a-b|}{|b|} \le 10^{-5}$.

\noindent
\begin{tabular}{| l | l | p{12cm} |}
  \hline
  Grupp & Poängvärde & Gränser \\ \hline
  $1$   & $15$       & Linnea har bara glömt en siffra \\ \hline
  $2$   & $85$       & Inga ytterligare begränsningar  \\ \hline
\end{tabular}

\section*{Förklaring av exempelfall}
I första exemplet vet hon alla siffror, därmed blir sannolikheten att hon gissar koden rätt $1$.
I andra exemplet är en siffra okänd, därmed är det $0.1$ chans att hon gissar koden rätt.
