\problemname{Flygande fika}
Joshua är törstig och har hört att Olles föräldrar inte är hemma. Han måste dock köra förbi $N$ stycken städer och hämta blommor. Han börjar på den första staden och
vill nå den sista. Räkna ut bensin bla bla. Ni löser flavor, bara för att visa ungefär hur det ska gå (nämn $N$ städer, relevant information och flavor text. ni vet redan)


\section*{Indata}
Den första raden innehåller ett heltal: $2\leq N \leq 10^6$ -- antalet städer. 
Därefter följer $N$ rader med två tal, $-10^9 \leq x \leq 10^9$, $-10^9 \leq y \leq 10^9$ positionen på städerna.


\section*{Utdata}
Skriv ut ett heltal -- minsta mängden bensin som krävs för att göra resan. Det är garanterat att svaret får plats i ett 32-bitars tal.

\section*{Poängsättning}
Din lösning kommer att testas på en mängd testfallsgrupper.
För att få poäng för en grupp så måste du klara alla testfall i gruppen.

\noindent
\begin{tabular}{| l | l | p{12cm} |}
  \hline
  Grupp & Poängvärde & Gränser \\ \hline
  $2$   & $15$       & $N=2$ \\ \hline
  $1$   & $25$       & $Y=0, 0 \leq X$ och x är stigande \\ \hline
  $4$   & $60$       & Inga ytterligare begränsningar  \\ \hline
\end{tabular}

\section*{Förklaring av exempelfall}
I första exemplet behöver .... färdas 7 steg från hus 1 till hus 2, och 8 steg från hus 2 till 3 hus, vilket blir 15 steg sammanlagt.
